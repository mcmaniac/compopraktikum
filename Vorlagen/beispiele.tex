% Falls sich herrausstellt, dass ein Bild an einer sehr ungünstigen
% Stelle ist (z.B. ein Theoriebild im Aufbauteil) ist [htbp]
% (bedeutet: here, top, botoom) in [!ht] zu ändern, wenn das bild nach
% oben soll, und [!hb] wenn es weiter nach unten soll

% Figure Vorlage
\begin{figure}[htbp]
  \centering
  \includegraphics[width=1\textwidth]{img/}
  \caption{}
  \label{img}
\end{figure}


% Zwei Bilder auf eine Seite
% Schmale Bilder brauchen dabei width<1, mit 1 geht das bis zu einem Verhältnis von Breite/Höhe von 4/3
\begin{figure}[htbp]
  \centering
  \subfigure[]{\includegraphics[width=1\textwidth]{img/}\label{img.}}
  \subfigure[]{\includegraphics[width=1\textwidth]{img/}\label{img.}}
  \caption{}
  \label{img.}
\end{figure}

% Zwei Bilder auf eine Höhe
\begin{figure}[htbp]
  \centering
  \subfigure[]{\includegraphics[width=0.45\textwidth]{img/}\label{img.}}
  \subfigure[]{\includegraphics[width=0.45\textwidth]{img/}\label{img.}}
  \caption{}
  \label{img.}
\end{figure}

% Matrix
\begin{pmatrix} 
  a_1 & a_2 & a_3 \\
  b_1 & b_2 & b_3 \\
  c_1 & c_2 & c_3  
\end{pmatrix} 

% Fallunterscheidung
